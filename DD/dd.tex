\documentclass[a4paper]{report}
\usepackage[T1]{fontenc}
\usepackage[utf8]{inputenc}
\usepackage[english]{babel}
\usepackage{titlesec}
\usepackage{lipsum}
\usepackage{booktabs}
\usepackage{hyperref}
\usepackage{graphicx}
\usepackage{float}
\usepackage{rotating}
\usepackage[dvipsnames]{xcolor}

\usepackage{geometry}
\usepackage{pdflscape}
\usepackage{caption}
\graphicspath{{./img/}}

\begin{document}

%%The two following lines remove the line "Chapter n" at the beginning of each chapter, before the title
%\titleformat{\chapter}[display]
%  {\normalfont\bfseries}{}{0pt}{\Large}
\titleformat{\chapter}[hang] 
{\normalfont\huge\bfseries}{\thechapter}{1em}{} 

\title{SafeStreets}
\author{Rosetti Nicola, Sartoni Simone, Torri Vittorio}
\date{date} %%TODO

\begin{titlepage}
\begin{figure}[t]
\centering
\includegraphics[width=0.3\textwidth]{Logo}
\end{figure}
\begin{center}
    \textsc{ \LARGE{Politecnico di Milano \\}}
	\textsc{ \Large {School of Industrial and Information Engineering\\ }}
	\textnormal{ \Large{Master of Science in Computer Science and Engineering\\}}
	\vspace{3mm}
	\textnormal{ \Large{Software Engineering 2 Mandatory Project\\}}
	\vspace{30mm}
	\fontsize{10mm}{7mm}\selectfont 
    \textup{Safe Streets.}\\
    \textnormal{ \LARGE{Design Document\\}}
\end{center}

\vspace{18mm}

\begin{center}
    \textnormal{\large{\bf Authors:\\}}
	{\large Rosetti Nicola \\ Sartoni Simone \\ Torri Vittorio }
	\fontsize{10mm}{5mm}\selectfont 
\end{center}
\vspace{15mm}

\centering{\large{
Academic Year 2019/2020 \\
\vspace{10mm}
Milano, 09/12/2019 \\
\vspace{2mm}
Version 1.0 
}}

\end{titlepage}

\newgeometry{top=3cm}
\tableofcontents
\listoffigures
\begingroup
\let\clearpage\relax %avoid to put it on a new page
\listoftables
\endgroup
\restoregeometry

\chapter{Introduction}
\section{Purpose}
\section{Scope}
\section{Definitions,Acronyms,Abbreviations}
\section{Revision History}
\section{Reference Documents}
\section{Document Structure}

\chapter{Architectural Design}
\section{Overview: High­‐level components and their	interaction}
The application will be developed using the client-server paradigm on a three-tiered architecture. The three layers of the application (Presentation, Application and Data) are divided into clusters of machines (i.e. tiers) that actually cooperate to provide a specific functionality. In this case we have three tiers  and each tier is responsible for one of the three layers. The client side is responsible (only) for the presentation layer; therefore, in this architecture, the thin-client has been adopted considering the fact that the required functionalities client-side are limited. The UIs provided are just meant to show results and to allow clients to choose what they want.  In the App case, the client contains all the presentation layer while in the WebApp case the layer is splitted between the client and WebServer; the WebServer is responsible for contacting the application server and forward the client requests to it. The Application tier takes care of the application layer encapsulating all is needed concerning the application logic. It receives the requests from the clients and handles them. It's also responsible for sending asynchronous notifications to the presentation layer when certain conditions are met.  Here we have multiple Servers cooperating together to improve performance, scalability, fault tolerance and availability. An elastic component (i.e. load balancer) is used to rule the accesses to different Application Servers, dinamically balancing the load among all the Servers. The Application tier communicates with the Data tier, responsible for the Data Access layer. This tier is composed by several DataServers: each one is associated with a single replica of the data and exploits the DBMS technology to access the DataBase. The Database is fully replicated in different nodes. Techniques and protocols are used to ensure consistency among replicas: they will be fullt explained in "Other design decisions" section. Again a load balancer is used to dinamically share the load among different machines. 
To ensure and improve security firewalls are installed before and after the application servers to filter accesses from external and unsafe networks. By the creation of a DMZ (demilitarized zone) external entities can only have access to the exposed services. Security is crucial because the application works mainly with sensible informations.
To provide the required functionalities the system exploits datawarehousing. The datawarehouse is a component in the Data tier able to deal with historical data and aggregate data taken from the Databases exploiting data mining technologies to answer complex queries: used techniques are clustering, associative rules and classification. This component periodically queries one dataServer to retrieve new information and updates on the data since its last update. 
\section{Component view}

\section{Deployment view}
\section{Runtime view}
\section{Component interfaces}
\section{Selected architectural	styles and patterns}
\section{Other design decisions}
\subsection{Consistency and update strategies among replicas}
In this application,  we need a client-centric consistency among replicas because end users (and therefore application servers that act as clients towards the DataServers) don't always connect to the same DataServer, due to the presence of the load balancer. Every DataServer can respond to a request to read or write data so we use an active replication protocol. In particular we exploit leaderless replication in which the decision on the value to read and the write to perform is decided by all the replicas or at least a quorum of them. The type of chosen consistency model  is the "read your writes": the effect of a write operation by a process on a data item x will always be seen by a successive read operation on x by the same process. 
Concerning the update propagation we opted for propagating a notification of the operation, assuming that there will be more writes than reads.  The propagation strategy chosen is the Gossiping strategy: when a replica is updated then it just propagates that update to all the nodes that it knows; if a replica receives an update that it has already received then the probability of propagating that information is decreased on that replica. 

\chapter{User interface design}

\chapter{Requirements Traceability}

In this section is shown how the requirements are actually ensured and what components actually ensure them. It's worth to notice that for the sake of simplicity the Router has been ignored but it's always involved when dealing with a request coming from one of the clients. 
\begin{itemize}
\item {[R1]}	\label{R1}The system must allow people to register to it providing personal data (name, surname, birthdate, identity card number, fiscal code) and selecting a username and a password: This requirement is provided by the AccessManager that actually handles requests of registration from the client.
\item {[R2]}	\label{R2}The system must verify the correctness of the provided personal data of a registered user checking them from the identity card number, blocking the registration if they are not correct: The AccessManager checks the received data from the Client and checks them calling the munipality API services that actually possesses sensible data of the users. /
\item {[R3]}	\label{R3}The system must allow registered users to login through their username and password: this requirement is provided by the AccessManager (AuthoritiesLogin and CitizenLogin components) that handles the requests of login from the user. 
\item {[R4]}	\label{R4}The system must allow logged user to fill a report violation form: this requirement is fulfilled by the User Mobile App comonent that allows user to choose the "Report a violation" feature.
\item {[R5]}	\label{R5}The system must let the user select the type of violation detected: the requirement is fulfilled by the User Mobile App component, showing an empty space to fill with the type of violation detected in the report a violation process.
\item {[R6]}	\label{R6}The system must allow the user to insert the license plate in a violation report:  the requirement is fulfilled by the User Mobile App component, showing an empty space to fill with the license plate of the car that committed the violation in the report a violation process.
\item {[R7]}	\label{R7}While reporting the violation, the system must allow users to take one or more pictures of the potential violation: the requirement is fulfilled by the User Mobile App component, allowing user to click on the "Take a picture" button.
\item {[R8]}	\label{R8}The system must not allow users to choose pictures not taken in the moment of the report: this requirement is guaranteed by the User Mobile App by not allowing client to just pick some random picture from his local storage.
\item {[R9]}	\label{R9}The system must collect the current position of the user, using GPS: the User Mobile App component ensures this requirement by taking the current position of the User using GPS location.
\item {[R10]}	\label{R10}The system must allow user to confirm or delete the current report: the User Mobile App component ensures this requirement by either sending the report for a check through the "confirm" button or deleting the current report clicking on the back button.
\item {[R11]}	\label{R11}After confirmation, the system must add the current date and time to the report: the ReportManager actually takes the requests from the client and attaches to them the current time and date taken from the application server internal clock.
\item {[R12]}	\label{R12}The system must store confirmed report: the ReportManager is responsible for this requirement, calling the DataServer to actually store the report.
\item {[R13]}	\label{R13}The system must check reports to try to find if the pictures of the violations have been modified: the ReportManager manages to check if the pictures are fake or not or have been modified.
\item {[R14]}	\label{R14}The system must try to find, according to the GPS position of the user and the pictures sent, if the position is fake or not: the ReportManager ensures this requirement by running an algorithm to define if the position is correct or not. 
\item {[R15]}	\label{R15}The system must discard the report if it has been recognized as fake according to the previous requirements (R13-R14): the ReportManager is in charge of this, deleting the report if recognized as fake.
\item {[R16]}	\label{R16}The system must try to automatically recognize the license plate in the photo, possibly with the help of the value inserted by the user: the Report Manager is in charge of this , running an algorithm for text recognition in the picture(s).
\item {[R17]}	\label{R17} ????? The system must store into stable memory the reported violation if correct (i.e. not recognized as fake). ??????
\item {[R18]}	\label{R18}The involved municipality must be calculated considering in which city the reported violation has been found, based on the GPS position of the user that has sent the report: the reportManager provides this requirement by asking the Maps Service the municipality associated to the position of the user.
\item {[R19]}	\label{R19}The system must send the reported violations to the involved municipality: the ReportManager and the the NotificationManager are responsible for this. After having saved it, the ReportManager sends the report to the NotificationManager that forwards it to the involved municipality.
\item {[R20]}   \label{R20}The system must allow an agent to see the reports for its municipality, checking them in order of arrival: The Municipality Web app component is responsible for this, showing on screen the reports in order of arrival.
\item {[R21]}	\label{R21}The system must allow an agent to issue a traffic ticket to a certain person (i.e. license plate) through the correspondent municipality service:
\item {[R22]}	\label{R22}The system must allow an agent to put on hold a violation report if it needs to be checked in person: The MunicipalityAgent Web App ensures this by shwoing the possibility of putting a report on hold while checking it.
\item {[R23]}	\label{R23}The system must allow an agent to discard a violation if it has been verified as fake or it cannot be verified (the vehicle is not there anymore) or it is a duplicated report.??
\item {[R24]}	\label{R24}The system must allow an agent to retrieve the data of the author of a violation report: the MunicipalityAgent Web App asks to the Municipality API Data about the author of the report. 
\item {[R25]}	\label{R25}The system must allow an agent to create an account, asking the municipality services to verify its identity: the AccessManager is responsible for this, asking the Municipality API to check the identity of the agent.
\item {[R26]}	\label{R26}The system must allow an agent to login, inserting its username and password: Municipality Web App and AccessManager are responsible for this. The Municipality Web App allows user to insert data, while AccessManager checks if they are correct.
\item {[R27]}	\label{R27}The system must mine this information from the reported violations.
\item {[R28]}	\label{R28}The system must allow the users (even if not authenticated) to select the see information for a city. The user can choose either the city where he is, using the GPS position, or an arbitrary selected location. The Mobile App component ensures this requirement by allowing user to choose the city he wants.
\item {[R29]}	\label{R29}The system must allow the user to select information about streets or areas in the city selected and to specify if he wants information for a specific street or area or a classification of streets or areas. The Mobile Web App is responsible for this requirement by allowing user to choose the area or street he wants.
\item {[R30]}	\label{R30}The system must show the the data corresponding to the selection of [R29]: the mobile Web App and Statisticsmanager components are responsible for this requirement; the StatisticsManager contacts the MapAPI to provide an image for the city/area/street selected. This image is then shown by the Mobile App component to the user.
\item {[R31]}	\label{R31}The system must take information about accidents and tickets from the municipality. //TODO
\item {[R32]}	\label{R32}The system must use this information to build statistics, crossing them with reported violations: the StatisticsManager and the DataWareHouse component ensure this requirement; the DataWareHouse components build 
\item {[R33]}	\label{R33}The system must not allow common users to see confidential data about other people: The Mobile App component is in charge for this requirement by showing only data that does not violate privacy.
\item {[R34]}	\label{R34}The system must allow the user to choose a topic: areas or streets with most accidents, areas or streets with the highest number of traffic tickets issued, areas or streets where there have been the best improvements, information for a specific area or street. The Mobile App iprovides this requirement by allowing user to choose the topic is interested into.
\item {[R35]}	\label{R35}The system must show to the user the information about the topic selected according to [R34]: The Mobile App component ensures this requirement by showing as an image the topic requested to the user.
\item {[R36]}	\label{R36}The system must allow authenticated supervisors to retrieve information about the vehicles with the highest number of violations in a selected area or street: the checkStatistics and Municipality //TODO 
%\item {[R37]}	\label{R37}The system must allow supervisors to see what areas or streets has seen the best improvements in the ticket emissions.
\item {[R37]}	\label{R37}The system must allow supervisors to access only information about their own municipality: the MunicipalitySupervisor Web App  is responsible for this, allowing supervisor to only check Statistics and suggestions about his city. 
\item {[R38]}	\label{R38}The system must allow a supervisor to create an account, asking the municipality services to verify its identity: the AccessManager and the MunicipalitySupervisor Web App are responsible for this requirement. the accessManager receives the request and ask the MunicipalityAPI to check if the inserted identity is correct.
\item {[R39]}	\label{R39}The system must allow a supervisor to login, inserting its username and password: The MunicipalitySupervisor Web App and the AccessManager are responsible for the requirement. The MunicipalitySupervisor WebApp allows supervisor to enter his username and password and forwards the request to the AccessManager that handles it asking the DataServer if they are correct.
\item {[R40]}	\label{R40}The system must take information about accidents, tickets and street networks (bike lanes, sidewalks, parking areas,…) from the municipality, exploiting the municipality services: the DataWareHouse component ??'
\item {[R41]}	\label{R41}The system must elaborate this information, combined with reports information, and try to find possible solutions for problems: the DataWareHouse component is responsible for this requirement trying to aggregate data coming from the Databases concerning reports, issued traffic tickets and accidents.
\item {[R42]}	\label{R42}The system must notify the municipality about new possible interventions: the NotificationManager the DataWareHouse and SuggestionManager components are responsible for this requirement: the DW component forwards the new intervention to the SuggestionManager that forwards it to NotificationManager that sends it to  the involved municipality.
\end{itemize}

\chapter{Implementation, integration and test plan}

\chapter{Effort spent}

\begin{table}[H]
\centering
Nicola Rosetti \\
%\begin{tabular}{lll}
\begin{tabular}{p{2cm}p{1.5cm}p{7cm}}
\toprule
\textit{Date} & \textit{Hour} & \textit{Section} \\ \midrule
17-10-2019 & 1.5 h* & Goals \\ \midrule
\bottomrule
\end{tabular}
\caption[Nicola Rosetti's effort table]{}
\end{table}


\vspace*{1 cm}
\begin{table}[H]
\centering
Simone Sartoni \\
\begin{tabular}{p{2cm}p{1.5cm}p{7cm}}
\toprule
\textit{Date} & \textit{Hour} & \textit{Section} \\ \midrule
17-10-2019 & 1.5 h* & Goals \\ \midrule
\bottomrule
\end{tabular}
\caption[Simone Sartoni's effort table]{}
\end{table}
\vspace*{1 cm}
\begin{table}[H]
\centering
Vittorio Torri \\
\begin{tabular}{p{2cm}p{1.5cm}p{7cm}}
\toprule
\textit{Date} & \textit{Hour} & \textit{Section} \\ \midrule
18-11-2019 & 1 h* & Goals \\ \midrule
\bottomrule
\end{tabular}
\caption[Vittorio Torri's effort table]{}
\end{table}
\textit{* Group work}

\chapter{References}


\end{document}
